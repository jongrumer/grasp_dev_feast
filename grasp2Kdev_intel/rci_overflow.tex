\documentclass[a4paper,titlepage]{article}
\usepackage{longtable}
%\usepackage{ams}
\usepackage{amsmath,amssymb,latexsym,theorem}
\usepackage{bm}
\usepackage[dvips]{color}
\begin{document}
\title{Notes on improvements in RCI} 
\author{Per J\"onsson\\
The Compas Group}  
\date{\today}
\maketitle

\section{Memory allocation and overflow}
For large cases the scalar version of \verb+rci+ fails to allocate memory for the Davidson diagonalization. The reason is an overflow in integer variables.\medskip\\
The fix would be to redefine all relevant integer variables to \verb+integer*8+. In the first round we look for the following variables:
\begin{verbatim}
     nelmnt
     nelmnt_a
     nelmnttmp
     nelmntt
     nstore
\end{verbatim}
There may be more variables and this needs careful attention.
\subsection*{Libraries}
In \verb+lib92+ we have \verb+nelmnt+ in:
\begin{verbatim}
     spicmv2.f:      COMMON/HMAT/PNTEMT,PIENDC,PNIROW,NELMNT
     spicmv.f:      COMMON/HMAT/PNTEMT,PIENDC,PNIROW,NELMNT
\end{verbatim}
In \verb+mpi+ we have \verb+nelmnt+ in:
\begin{verbatim}
     spicmvmpi.f:      COMMON/HMAT/PNTEMT,PIENDC,PNIROW,NELMNT
\end{verbatim}
\subsection*{Applications, rci}
In \verb+ric+ we have \verb+nelmnt+, \verb+nelmnt_a+, \verb+nelmnttmp+, \verb+nelmntt+,
\verb+nstore+ in:
\begin{verbatim}
     dnicmv.f:      COMMON/HMAT/PNTEMT,PIENDC,PNIROW,NELMNT
     
     genmat2.f:      SUBROUTINE genmat2 (irestart, nelmnt_a, elsto)
     genmat2.f:*   The mpi version (genmat2mpi) also gets nelmnt_a and elsto
     genmat2.f:     &  NCOREtmp, NVPItmp, NKEItmp, NVINTItmp, NELMNTtmp, NCFtmp
     genmat2.f:      NELMNT_a = NELMNTtmp
     genmat2.f:      DENSTY = DBLE (NELMNTtmp) / DBLE ((NCFtmp*(NCFtmp+1))/2)
     genmat2.f:      WRITE (24,312) NELMNTtmp  % FORMAT STATEMENT NEEDS ATTENTION
		
     genmat.f:      COMMON/HMAT/PNTEMT,PIENDC,PNIROW,NELMNT
     genmat.f:     &  NCOREtmp, NVPItmp, NKEItmp, NVINTItmp, NELMNTtmp, NCFtmp
     genmat.f:      nelmnt = 0     ! Counting continues in setham
     genmat.f:! nelmnt, eav, elsto obtained (to be further modified in setham)
     genmat.f:               nelmnt = nelmnt + nelc
     genmat.f:         CALL setham (myid, nprocs, jblock, elsto, icstrt, nelmnt   
     genmat.f:         NELMNTtmp = NELMNT
		
     hmout.f:      COMMON/HMAT/PNTEMT,PIENDC,PNIROW,NELMNT
		
     maneig.f:      COMMON/HMAT/PNTEMT,PIENDC,PNIROW,NELMNT
     maneig.f:!            IF (NELMNT .LT. NBRKEV) THEN
     maneig.f:               NSTORE = NELMNT+NELMNT/2+(NCF+1)/2
     maneig.f:         print *, 'nelmnt = ', nelmnt
     maneig.f:                  CALL ALLOC (PNTEMT,NELMNT,8)
     maneig.f:                  CALL ALLOC (PNIROW,NELMNT,4)
		
     matrix.f:     :      /HMAT/PNTEMT,PIENDC,PNIROW,NELMNT
     matrix.f:*           ...eav, nelmnt are also obtained from genmat
     matrix.f:            CALL genmat2 (irestart, nelmnt_a, elsto)
     matrix.f:         CALL genmat2 (irestart, nelmnt_a, elsto)
		
     setham_gg.f:      SUBROUTINE SETHAM (myid, nprocs, jblock, ELSTO,ICSTRT, nelmntt
     setham_gg.f:     :      /HMAT/PNTEMT,PIENDC,PNIROW,NELMNT
     setham_gg.f:     &  NCOREtmp, NVPItmp, NKEItmp, NVINTItmp, NELMNTtmp, ncftmp
     setham_gg.f:      nelmnt = nelmntt                           
     setham_gg.f:         NELMNT = NELMNT + NELC
     setham_gg.f:      NELMNTtmp = NELMNT
\end{verbatim}
We need to pay careful attention to all this and also see if there are integers defined that in turn require other changes. 
\subsection*{Applications, rscf}
In \verb+rscf+ we have \verb+nelmnt+  in:
\begin{verbatim}
     hmout.f:      COMMON/HMAT/PNTEMT,PIENDC,PNIROW,NELMNT
		
     maneig.f:     :      /HMAT/PNTEMT,PIENDC,PNIROW,NELMNT
		
     matrix.f:      COMMON/hmat/pntemt,piendc,pnirow,nelmnt
     matrix.f:      READ (30) nelmnt
     matrix.f:      CALL alloc (pnirow, nelmnt, 4)
     matrix.f:      CALL alloc (pntemt, nelmnt, 8)
     matrix.f:      DO i = 1, nelmnt
     matrix.f:     &           (irow(i), i = 1, nelmnt)
		
     setcof.f:      COMMON/HMAT/PNTEMT,PIENDC,PNIROW,NELMNT
     setcof.f:         READ (30) NELMNT
     setcof.f:         CALL alloc (PNIROW, NELMNT, 4)
     setcof.f:     &             (IROW(I), I = 1, NELMNT)
     setcof.f:            READ (30) NELMNT
     setcof.f:            CALL alloc (PNIROW, NELMNT, 4)
     setcof.f:     &                (IROW(I), I = 1, NELMNT)
		
     setham.f:      COMMON/HMAT/PNTEMT,PIENDC,PNIROW,NELMNT
     setham.f:            IF (LOC .GT. NELMNT) THEN
     setham.f:               PRINT *, '  LOC = ', LOC, '  NELMNT = ', NELMNT
     setham.f:            IF (LOC .GT. NELMNT) THEN
     setham.f:               PRINT *, '  LOC = ', LOC, '  NELMNT = ', NELMNT
\end{verbatim}
\section{Implemented changes}
The changes have been implemented at Monster in \verb+/home/per/programs/grasp2k_light_2014-02-12+
\section{Breit integrals}
The computation of Breit integrals have been identified as a bottleneck for scalar, and even more so for parallel, \verb+rci+ calculations.
The computation of Breit integrals are inside the double loop over CSFs in \verb+setham+ by means of calls to \verb+brint1+, ..., \verb+brint6+. 
For every interaction between two CSFs there is a call to \verb+brint1+ that loops through the full integral list. If the integral is not available then: 
\begin{enumerate}
\item space is allocated to keep the integral
\item the integral is computed
\item the integral is stored
\end{enumerate}
When the integral list is long this takes forever.\medskip\\
We intend to recode this so that all the Breit integrals are computed in advance as is done for the Rk integrals. This is a two-step procedure implemented in \verb+genintrk+:
\begin{enumerate}
\item loop through the orbital set and count the maximum number of integrals
\item allocate space for all the integrals
\item loop through the orbital set, compute and store the integrals
\end{enumerate} 

\end{document}



